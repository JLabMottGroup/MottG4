\documentclass{beamer}
\usetheme{Warsaw}
\useoutertheme{miniframes}

\usepackage[utf8x]{inputenc}
\usepackage{default}
\usepackage{amsmath}
\usepackage{amssymb}
\usepackage{amsfonts}
\usepackage{graphicx}
\usepackage{enumerate}

\title{\textbf{GEANT4 Simulation of the Jlab MeV Mott Polarimeter}}
\author{Martin McHugh\\
		 mjmchugh@jlab.org}
\date{2015-07-01}

\begin{document}

\begin{frame}
\maketitle
\end{frame}

\begin{frame}
\frametitle{The Problem}
\begin{columns}[c] % the "c" option specifies center vertical alignment
\column{.5\textwidth} % column designated by a command
We don't know the form of the effective Sherman function for targets of finite thickness, $S(d)$.
\column{.5\textwidth}
\includegraphics[scale=0.4]{Plots/DataAsymVsThickness.pdf}
\end{columns} 
\end{frame}

\begin{frame}
 \frametitle{How Can Simulation Help Us?}
 \begin{columns}[c] % the "c" option specifies center vertical alignment
 \column{.5\textwidth} % column designated by a command
  \begin{itemize}
   \item Allows us to examine contributions to detector signal individually.
   \item Beam is treated as 100\% polarized in $y$ direction.
   \item Gaussian, circular beam profile with width of 1 mm.
  \end{itemize}
 \column{.5\textwidth}
  \textbf{Why Brute Force doesn't work:} 1 $\mu$A is 6.24$\times 10^{12}$ $e^{-}/s$ and we need $\approx$ 1000 $\mu$As of data for a decent measurement. Can only simulate 100 million events per day...
 \end{columns}
\end{frame}

\begin{frame}
 \frametitle{How to Generate Single-Scattering Events}
 \begin{columns}[c] % the "c" option specifies center vertical alignment
  \column{.5\textwidth} % column designated by a command
   \begin{enumerate}
    \item Pick point $\vec{x}_1$ in the beam profile on the target.
    \item Calculate energy loss to $\vec{x}_1$. Get new energy $E_1$.
    \item Pick point $\vec{x}_2$ in acceptance to throw at.
    \item Calculate $\sigma(\theta_1,\phi_1, E_1)$ based on $\vec{x}_1$, $\vec{x}_2$.
    \item Throw random number, $x$. If $ x < \sigma $ throw electron. Else, repeat from 1.
   \end{enumerate}
  \column{.5\textwidth}
   \includegraphics[scale=0.4]{Plots/DataSimAsymVsThickness.pdf}
 \end{columns}
\end{frame}

\begin{frame}
 \frametitle{How to Generate Double-Scattering Events}
  \begin{enumerate}
   \item Pick point $\vec{x}_1$ in the beam profile on the target.
   \item Calculate energy loss to $\vec{x}_1$. Get new energy $E_1$.
   \item Pick point $\vec{x}_2$ in target with $|\vec{x}_2 - \vec{x}_1| < r_E.$
   \item Calculate $\sigma_1(\theta_1,\phi_1, E_1)$ based on $\vec{x}_1$, $\vec{x}_2$.
   \item Calculate energy loss to $\vec{x}_2$. Get new energy $E_2$.
   \item Pick point, $\vec{x}_3$, in acceptance to throw at.
   \item Calculate $\sigma_2(\theta_2,\phi_1, E_2)$.
    \item Throw random number, $x$. If $ x < \sigma_1\sigma_2 $ throw electron. Else, repeat from 1.
  \end{enumerate}
\end{frame}
 
\begin{frame}
 \frametitle{Double Scattering Asymmetry}
  The Asymmetry is calculated to be:
  \begin{equation*}
   A_{d.s.} = \frac{L-R}{L+R} = -01.05\% \pm 0.06\%
  \end{equation*}
  for all target thicknesses. The problem now becomes one of determining how much of a dilution this is at each target thickness.
\end{frame}
 
\begin{frame}
 \frametitle{Scattering into Left Detector}
  \centering
   \includegraphics[width=0.9\linewidth]{Plots/Left_2_Scattering.pdf}
\end{frame}

\begin{frame}
 \frametitle{Double Scattering Generation: Method 2}
 \begin{enumerate}
  \item Pick a scattering position, $\vec{x}_1$, within the intersection of the beam and our target.
  \item Pick a direction direction, $(\theta_1,\phi_1)$, from the uniform unit sphere.
  \item Pick a point, $\vec{x}_2$, uniformly between $\vec{x}_1$ and the edge of the foil (or 0.16 mm as in the previous example).
  \item Pick a point, $\vec{x}_3$, in the acceptance the primary collimator.
  \item Throw from $\vec{x}_2$ towards $\vec{x}_3$.
\end{enumerate}
Not weighting by cross section allows for an easier integral in the rate calculation. 
\end{frame}


\begin{frame}
 \frametitle{Calculating Rates}
  In order to compare both types of simulation and to compare simulation to data, we must be able to calculate rates. The rate is given as
  \begin{equation*}
   \mathcal{R} = \mathcal{L} \int_x \sigma(x)\epsilon(x)
  \end{equation*}
  where $\epsilon$ is the effective acceptance of the detectors and $x$ are the degrees of freedom over which the integral is performed.
\end{frame}

\begin{frame}
 \frametitle{Calculating Rates (Single Scattering)}
 In the case of single scattering the integral can be simplified to:
 \begin{align*}
  \mathcal{R} &\approx \mathcal{L} \langle\sigma\rangle \frac{N_{hit}}{N_{thrown}}\Delta\cos\theta\Delta\phi
 \end{align*}
 \begin{table} [h!]
 \centering
 \begin{tabular}{| c | c | c | c | c |} 
  \hline d ($\mu$m) & $\mathcal{R}_L$ (Hz/$\mu$A) & $\mathcal{R}_R$ (Hz/$\mu$A) & $\mathcal{R}_\mathrm{avg}$ & $\mathcal{R}_\mathrm{data}$ \\
  \hline 0.05 & 5.3 & 11.6 & 8.5 & 9.3\\ 
  \hline 1.00 & 104.3 & 241.9 & 173.1 & 214.3 \\
  \hline
 \end{tabular}

\end{table}
 This is a good sanity check.  
\end{frame}

\begin{frame}
 \frametitle{Calculating Rates (Double Scattering)}
 \begin{itemize}
  \item Attempting the same simplification doesn't work in the double scattering case. Rates are $\approx10^{-12}$ smaller than the single scattering case. 
  \item More thought needs to go into performing this integral in order to make the simulation work. 
  \item Once this works, we should be able to calculate
   \begin{equation*}
    A(d) = \frac{\left[\mathcal{R}_{L_1}(d)-\mathcal{R}_{R_1}(d)\right] + \left[\mathcal{R}_{L_2}(d)-\mathcal{R}_{R_2}(d)\right]}{\left[\mathcal{R}_{L_1}(d)+\mathcal{R}_{R_1}(d)\right] + \left[\mathcal{R}_{L_2}(d)+\mathcal{R}_{R_2}(d)\right]}
   \end{equation*}
   directly from simulation.
 \end{itemize}
\end{frame}

\begin{frame}
 \frametitle{Spectra}
  \includegraphics[scale=0.54]{Plots/SimulatedSpectra.pdf}
\end{frame}

\begin{frame}
 \frametitle{Summary}
 \begin{itemize}
  \item Single scattering simulation gives good results but no $d$-dependence in Asymmetry.
  \item Can't calculate rate yet for Double-scattering. Asymmetry is small. Need to determine proper dilution.
  \item Spectra look decent, can't tell how much but it looks like the double scattering will influence the low energy shoulder to some degree.
 \end{itemize}
\end{frame}

\begin{frame}
 \frametitle{Energy Loss in the Gold Foils}
 \begin{columns}[c] % the "c" option specifies center vertical alignment
  \column{.5\textwidth} % column designated by a command
   Using the table at right, we determine the linear fit
   \begin{equation*}
    \frac{dE}{dx}(E) =  \frac{0.272}{\mathrm{mm}}\times E + 1.888 \frac{\mathrm{MeV}}{\mathrm{mm}}. 
   \end{equation*}
   Numerical integration of the above gives us energy loss within the target. Note: A particle with initial energy of 5 MeV will only lose ~3 keV/$\mu$m and will lose 500 keV in  $\approx\,200 \mu$m. 
  \column{.5\textwidth}
   \begin{tabular} {|c|c|}
    \hline 
    Energy [MeV] & dE/dx [MeV/mm] \\
    \hline 
    1.0 & 2.179 \\
    2.0 & 2.422 \\ 
    3.0 & 2.702 \\
    4.0 & 2.980 \\
    5.0 & 3.254 \\
    6.0 & 3.526 \\
    7.0 & 3.796 \\
    8.0 & 4.065 \\
    \hline
   \end{tabular}
   Data from NIST estar database.
 \end{columns}
\end{frame}

\begin{frame}
 \frametitle{What Steigerwald Did}
  His method of calculating multiple scattering's influence was direct integration of some form:
  \begin{equation*}
   N = \int\limits_{\theta = 0}^{\pi} \int\limits_{\phi = 0}^{2\pi} \int\limits_{x_1 = 0}^{D} \int\limits_{\psi=\theta_{2}}^{\theta_2+\Omega_\theta}\sigma_1(x_1,\theta, \phi)\sigma_2(x_2,\theta_2)E(x_1,x_2) d\psi dl d\phi d\theta. 
  \end{equation*}
  The problem is that his source for this integral is in German and his code is poorly documented and also in partial German.
\end{frame}
 
\end{document}
