\documentclass[11pt]{article}
\usepackage{amsmath}
\usepackage{graphicx}
\usepackage{layout}
\usepackage{gensymb}
\usepackage{caption} 
\usepackage{subcaption}
\usepackage{hyperref}

%Paper formatting stuff
\RequirePackage[letterpaper]{geometry}	%USLetter = standad paper.
\textheight = 615pt 
\textwidth = 470pt
\oddsidemargin = 0pt
\marginparwidth = 60pt
\marginparsep = 12pt
\marginparpush = 0pt %Not shown
\topmargin = 0pt
\headheight = 0pt
\headsep = 0pt 
\footskip = 36pt
\pagestyle{plain}

\title{\textbf{GEANT4 Simulation of the Jlab MeV Mott Polarimeter}}
\author{Martin McHugh\\
		The George Washington University\\
		mjmchugh@jlab.org}
\date{2015-07-01}
\begin{document}

\maketitle

\begin{abstract}
The simulation of the JLab Mott Polarimeter is at a place where it can accurately simulate single scattering and the detector response thereof. Results are shown for this case. Additionally, work is being done to simulate realistic double scattering. Preliminary results and methods attempted are shown. 
\end{abstract}

\section{JLab Polarimeter and Mott Scattering}

The MeV Mott Polarimeter is located in the Continuous Electron Beam Accelerator Facility (CEBAF) injector at Jefferson Lab (JLab). It is used to measure the transverse polarization of the electron beam in the 2 - 10 MeV energy range. The polarimeter measures the elastic scattering asymmetry of electrons incident on the nuclei of a thin target foil. The foils used include gold, silver, and copper and range in thickness from 100-10,000 \AA. The elastically scattered electrons from the target foil pass through an aluminum collimator which sets the scattering angle of 172.6$^\circ$ $ \pm$ 0.1$^\circ$ with a per quadrant solid angle of 0.18 msr. The scattered electrons pass through the collimator, then pass through the 0.05 mm thick aluminum window and into the detector packages. Each detector package contains two plastic scintillators connected to PMTs for readout: a 1 mm $\times$ 25.4 mm $\times$ 25.4 mm wafer scintillator, the $\Delta$ E detector, and a cylindrical 76.2 mm diameter, 63.5 mm long scintillator, the E detector, which functions as a stop detector and calorimeter with a 3\% energy resolution. All of these geometric peices are modelled in the \texttt{GEANT4} simulation and the precise description can implemented can be found in \texttt{MottDetectorConstruction.cc}.

\subsection{Single Mott Scattering}

The polarimeter functions by measuring the \textit{Mott scattering asymmetry}. Mott Scattering describes elastic electron-nuclear scattering. The differential cross-section can be written as
\begin{equation}
 \frac{d\sigma}{d\Omega}(\theta) = I(\theta)\left(1 + S(\theta)\vec{P}\cdot\hat{n}\right)
\end{equation}
where $\vec{P}$ is the incoming beam's polarization, $S(\theta)$ is known as the Sherman function and 
\begin{equation}
 \vec{n} = \frac{\vec{p}\times\vec{p}^{\,\prime}}{\left|\vec{p}\times\vec{p}^{\,\prime}\right|}
\end{equation}
where $\vec{p}$ ($\vec{p}^{\,\prime}$) is the incoming (outgoing) momentum of the electron. In the case of ideal single scattering we expect to measure an asymmetry,
\begin{equation}
 \label{eq:SimpleAsym}
 A = \frac{N_L-N_R}{N_L+N_R} = P_yS(\theta).
\end{equation}
However, we actually observe that the asymmetry depends on target thickness as is shown in Fig. \ref{fig:DataVsSingleSim}. While we have many empirical fits that we can use, The goal of the \texttt{GEANT4} simulation is to see what gives rise to the observed if it can be reproduced numerically. 

\begin{figure}[!h]
 \centering
 \includegraphics[width=0.5\textwidth]{Plots/DataSimAsymVsThickness.pdf}
 \caption{Comparison of measurements (blue crosses) with single scattering simulation (orange diamonds). Simulated results have been scaled to assume $P = 89\%$.} 
\label{fig:DataVsSingleSim}
\end{figure}

\subsection{Double Mott Scattering}
It is our assumption that the target thickness dependence of the Mott scattering asymmetry is the result of multiply scattered electrons within the target foil. Simulation of this effect requires us to track the polarization over multiple steps. A Mott scattered electron beam carries a new polarization. 
\begin{equation}
 \vec{P}^\dagger = \frac{\left(\vec{P}\cdot\vec{n}+S(\theta)\right)\vec{e}_1 + U(\theta)\vec{e}_2 + T(\theta)\vec{e}_3}{1+\vec{P}\cdot\vec{n}\,S(\theta)} 
\end{equation}
where
\begin{eqnarray}
 \vec{e}_1 &=& \vec{n} \\
 \vec{e}_2 &=& \vec{n}\times\vec{P} \\
 \vec{e}_3 &=& \vec{n}\times\left(\vec{P}\times\vec{n}\right)
\end{eqnarray}
and $U(\theta)$ and $T(\theta)$ are functions which measure the spin transfer probability of the scattering. Plots of the relevant scattering functions for a selection of typical energies can be seen in Fig. \ref{fig:ScatteringFunctions}. Those four plots are generated using the scattering calculations performed by Xavier Roca Maza on 2015-05-29. These calculations form the basis of Mott scattering physics in our simulation. One can observe that the JLab Mott Polarimeter was built at the point where the asymmetry is largest rather than at the point where the figure of merit (shown in Fig. \ref{fig:FoM}) is maximized. 

\begin{figure}[!h]
\begin{subfigure}{.5\textwidth}
  \centering
  \includegraphics[width=.8\linewidth]{Plots/Mott_Au_CrossSection.pdf}
  \label{fig:CrossSec}
\end{subfigure}
\begin{subfigure}{.5\textwidth}
  \centering
  \includegraphics[width=.8\linewidth]{Plots/Mott_Au_ShermanFunction.pdf}
  \label{fig:ShermanFunction}
\end{subfigure}\\
\begin{subfigure}{.5\textwidth}
  \centering
  \includegraphics[width=.8\linewidth]{Plots/Mott_Au_SpinT.pdf}
  \label{fig:SpinT}
\end{subfigure}
\begin{subfigure}{.5\textwidth}
  \centering
  \includegraphics[width=.8\linewidth]{Plots/Mott_Au_SpinU.pdf}
  \label{fig:SpinU}
\end{subfigure}
 \caption{Mott physics as a function of angle for typical polarimeter energies.} 
\label{fig:ScatteringFunctions}
\end{figure}

\begin{figure}[!h]
 \centering
 \includegraphics[width=0.5\textwidth]{Plots/Mott_Au_FoM.pdf}
 \caption{Figure of Merit, $I(\theta)S(\theta)^2$, for typical polarimeter energies.} 
\label{fig:FoM}
\end{figure}

\subsection{Previous Efforts}
In the past, Michael Steigerwald, performed a direct integration of the form:
\begin{equation}
N = \int\limits_{\theta = 0}^{\pi} \int\limits_{\phi = 0}^{2\pi} \int\limits_{x_1 = 0}^{D} \int\limits_{\psi=\theta_{2}}^{\theta_2+\Omega_\theta}\sigma_1(x_1,\theta, \phi)\sigma_2(x_2,\theta_2)E(x_1,x_2) d\psi dl d\phi d\theta.
\end{equation}
and obtained a resulting thickness dependence of the Sherman Function to be:
\begin{equation}
S(d) = S(0)\frac{1 + 0.00272d^{0.866}}{1 + 0.23d^{0.866} + 0.0739d + 0.0146d^2+ 0.00339d^3}.
\end{equation}
Unfortunately, this result has not been reproducible. A summary of his work can be found in \url{https://wiki.jlab.org/ciswiki/images/2/28/AIP570_935_2001.pdf}.

\section{Simulating Mott Scattering}

To begin our simulation, we must generate electrons in our apparatus to represent certain physical cases. Before we discuss the particular cases the simulation can model in the following sections, we look at the properties of our ``incident beam'' of electrons on the target. 

In all of the following cases, the electron beam is assumed to have an initial polarization. While the user may modify this, the standard assumption made is that the beam is 100\% polarized in the positive $y$ direction; $\vec{P}_1 = \hat{j}$. The incident electrons are assumed to have momentum entirely in the $z$ direction; $\vec{p}_1 = p \hat{k}$. Finally, the beam is assumed to have a circular, Gaussian profile on the target with a width of 1 $\mu$m. All of the exact methods can be found in \texttt{MottPrimaryGeneratorAction.cc}.

\subsection{Single Scattering}
To look at the detector response to electrons that undergo exactly one Mott scattering process in the target, we use the following algorithm:
\begin{enumerate}
 \item Pick a scattering position, $\vec{x}_1$, within the intersection of the beam and our target.
 \item Pick a point, $\vec{x}_2$, in the acceptance the primary collimator.
 \item Calculate $\frac{d\sigma}{d\Omega}(\vec{x}_1,\vec{x}_2)$.
 \item Rejection sample against this cross-section. If accepted, proceed to generate the event. If rejected repeat steps 1-3.  
\end{enumerate}
The simulations shown herein all assume 100\% polarization in the $y$ direction and look only at 5 MeV electrons. In order to measure the Mott asymmetry from single scattering simulations, we simply use Eq. (\ref{eq:SimpleAsym}):
\begin{equation}
 A = \frac{N_L-N_R}{N_L+N_R} = -0.513 \pm 0.0005
\end{equation}
Unfortunately, the results do not change with target thickness as shown in Fig. \ref{fig:DataVsSingleSim} (Note that the simulation results have been scaled to assume that $P = 89\%$). This confirms that single scattering is not adequate to explain the results of thick target foils. Typical spectra from this event generator can be seen in Fig. \ref{fig:SimSinglevsDouble}.

\begin{figure}[!h]
 \centering
 \includegraphics[width=\textwidth]{Plots/SimulatedSpectra.pdf}
 \caption{Simulated normalized single (double) scattering spectra in blue (red). It is clear that double scattering contributes non-trivially to both the peak and to the shoulder, however rate calculations need to be performed to quantify the contribution accurately.} 
\label{fig:SimSinglevsDouble}
\end{figure}

\subsection{Double Scattering: First Method}

The first method I've used to calculate the effect of double scattering is as follows,
\begin{enumerate}
 \item Pick a scattering position, $\vec{x}_1$, within the intersection of the beam and our target.
 \item Pick a point, $\vec{x}_2$, within the target, such that $|\vec{x}_2-\vec{x}_1| <$ 0.16 mm. Beyond this distance in Gold a 5 MeV electron will lose over 500 keV and no longer be of interest to us.    
 \item Calculate $\frac{d\sigma_1}{d\Omega_1}(\vec{x}_1,\vec{x}_2)$.
 \item Pick a point, $\vec{x}_3$, in the acceptance the primary collimator.
 \item Calculate $\frac{d\sigma_2}{d\Omega_2}(\vec{x}_2,\vec{x}_3)$.
 \item Rejection sample against this $\frac{d\sigma_1}{d\Omega_1}\frac{d\sigma_2}{d\Omega_2}$. If accepted, generate electron at $\vec{x}_2$ towards $\vec{x}_3$  If rejected repeat steps 1-5.
\end{enumerate}
This method produces an asymmetry of $\boxed{A = -0.0111 \pm 0.0005}$. Unfortunately this asymmetry also does not scale with target thickness. Simulated spectra from this generator can be seen in Fig. \ref{fig:SimSinglevsDouble}. Scattering vertex information from this method can be seen in Fig. \ref{fig:DoubleScattering}.

\begin{figure}[!h]
\begin{subfigure}{\textwidth}
 \includegraphics[width=0.9\linewidth]{Plots/Left_2_Scattering.pdf}
\end{subfigure}\\
\begin{subfigure}{\textwidth}
 \includegraphics[width=0.9\linewidth]{Plots/Right_2_Scattering.pdf}
\end{subfigure}
 \caption{Double scattering information for hits in each detector. From left to right in the top row, the initial cross-section, scattering angle, and azimuthal angle. The bottom row contains the same information for the second scattering.}
\label{fig:DoubleScattering}
\end{figure}

The second method attempted to model double scattering is as follows:
\begin{enumerate}
 \item Pick a scattering position, $\vec{x}_1$, within the intersection of the beam and our target.
 \item Pick a direction direction, $(\theta_1,\phi_1)$, from the uniform unit sphere.
 \item Pick a point, $\vec{x}_2$, uniformly between $\vec{x}_1$ and the edge of the foil (or 0.16 mm as in the previous example).
 \item Pick a point, $\vec{x}_3$, in the acceptance the primary collimator.
 \item Throw from $\vec{x}_2$ towards $\vec{x}_3$.
\end{enumerate}
This method does not rejection sample as the previous two do but it may allow for calculation of rates if the proper integral can be performed. 

\subsection{Calculating Rates}
The basic equation for linking the simulation with data applies to total scattering rate detected,
\begin{equation}
 \mathcal{R} = \mathcal{L} \int_x \sigma(x)\epsilon(x)
\end{equation}
where $\mathcal{R}$ is the scattering rate in the detector, $\sigma$ is the cross section, $\epsilon$ is the effective solid angle of the detector, $\mathcal{L}$ is the luminosity and, $x$ represents the various degrees of freedom over which the integral is done. The GEANT4 package is used to do the implicit integral over degrees such as target position and various external processes (transportation and scattering outside of the target etc.). In the case of single scattering, the above equation is approximated as:
\begin{equation}
 \mathcal{R} \approx \mathcal{L} \left\langle\frac{d\sigma}{d\Omega}\right\rangle_{hits} \Delta\phi \Delta\cos\theta \frac{N_{hit}}{N_{thrown}}
\end{equation}
To test the accuracy of this method, simulations were run using angular ranges of $\phi \in [-10^\circ,10^\circ]$ centered around the appropriate collimator hole (left or right) and $\theta \in [170^\circ,175^\circ]$. Results from runs of 100,000 events for each detector at both 0.05 $\mu$m and 1.0 $\mu$m targets are shown in Table \ref{table:SimulatedRates}.

\begin{table} [h!]
 \centering
 \begin{tabular}{| c | c | c | c | c |} 
  \hline d ($\mu$m) & $\mathcal{R}_L$ (Hz/$\mu$A) & $\mathcal{R}_R$ (Hz/$\mu$A) & $\mathcal{R}_\mathrm{avg}$ & $\mathcal{R}_\mathrm{data}$ \\
  \hline 0.05 & 5.3 & 11.6 & 8.5 & 9.3\\ 
  \hline 1.00 & 104.3 & 241.9 & 173.1 & 214.3 \\
  \hline
 \end{tabular}
 \caption{Simulated rates for single scattering in the left and right detectors for our thinnest and thickest typical foils. Data averaged from \url{https://wiki.jlab.org/ciswiki/images/4/47/Analysis_Slides_7_29.pdf}}
 \label{table:SimulatedRates}
\end{table}
It appears that single scattering accounts for $\approx 90\%$ of the signal observed from our thinnest target and $\approx 80\%$.
In the case of double scattering, it's not immediately apparent how to calculate the rates since the form of $\epsilon(x)$ is much more complicated.

\section{Conclusions}

The \texttt{GEANT4} simulation of the JLab MeV Mott Polarimeter is a useful tool. So far it has produced single scattering results which have an asymmetry consistent with measurements extrapolated to zero target thickness and rates which are within 10-20\% of the measured values. Work is ongoing in two areas: producing an event generator which can accurately predict the asymmetry of double scattered electrons, calculating rates for double scattered electrons to determine the extent to which they dilute the single scattering signal. 

\end{document}